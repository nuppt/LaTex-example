\documentclass[a4paper,twocolumn]{article}
% 文 档 类 型, 例 如 article,[]内 是 选 项, 比 如 a4paper 设 置 为 A4纸

%
% 这 里 开 始 是 导 言 区
%
\usepackage{ctex}
\usepackage{url}
\usepackage{amsmath}
\usepackage{authblk}
\usepackage{booktabs}
\usepackage{graphicx}
\usepackage[numbib]{tocbibind}
\graphicspath{{fig/}} % 设 置

\title{如何阅读一篇论文}
\date{Auguse 2, 2013}
\author{S. Keshav}
\affil{David R. Cheriton School of Computer Science, University of Waterloo
	\\Waterloo, ON, Canada
	\\keshav@uwaterloo.ca}

%
% 这里开始时正文区
%
\begin{document}
	\maketitle
	\begin{abstract}
		\emph{研究人员花费大量时间阅读研究论文。然而,因为研读论文的技巧很少被教授,导致了很多浪费的付出。本文概述了针对阅读研究论文的一种实用且有效的“三轮阅读法”,同时也描述了如何使用该方法做文献综述。}
	\end{abstract}
	
	\section{引言}
	研究人员出于各种各样原因阅读论文,比如:为准备会议或课程而回顾它们,或者在相关领域保持跟踪,或调研一个新的领域。一个研究人员每年需要花费数百小时研读论文。
	学会高效阅读论文很重要,但却很少被传授。因此,新进的研究生必须依靠自己尝试和试错进行学习。他们在这个过程中浪费了很多精力,而且常常陷入挫折。
	多年以来,我一直使用一种简单的“三轮阅读法”,来帮助自己避免在获得鸟瞰图之前淹没在论文细枝末节中。它允许我评估一系列论文的所需的时间。另外,我可以根据我的需求和时间调整评估论文的深度。
	
	\section{三轮阅读法}
	此方法的关键想法在于分三轮阅读一篇论文,而非上来就从头到尾的斟字酌句。每一轮阅读在上一轮基础上,完成特定目标:第一轮了解论文大意;第二轮了解论文的主要内容但非细枝末节;第三轮帮助你深入理解论文。
		\subsection{第一轮}
		第一论快速浏览,获得论文的鸟瞰信息。你可以决定是否需要更多轮的阅读。这一轮花费5 ~ 10分钟,包括以下步骤:
		
		\begin{enumerate}
			\item 仔细阅读标题,摘要,和引言
			\item 阅读章节标题,但略过所有其他内容
			\item 浏览数学内容(如果有的话)以确定潜在的理论基础
			\item 阅读结论
			\item 粗略地看一下参考文献,识别出你已经读过的文献
		\end{enumerate}
	
		在第一轮结束时,你应该能回答五个问题Cs:
		
		\begin{enumerate}
			\item Category(分类):这篇论文属于什么类型?是实证论文?现有系统的分析?研究原型的描述?
			\item Context(背景):这篇论文与哪些论文有关联?哪些理论基础对分析问题有用?
			\item Correctness(正确性):论文的前提假设是否有效?
			\item Contribution(贡献):这篇论文的主要贡献是什么?
			\item Clarity(清晰度):这篇论文写得好吗(表述清晰)?
		\end{enumerate}
	
		基于上述信息,你可能选择不再进一步阅读(并且不打印,因此节省纸张)。原因可能是你对论文不感兴趣,或者由于你对该领域不甚了解以致难以理解论文,也可能是作者引入无效的假设。第一遍阅读对于不在你研究领域但某天可能证明相关的论文已经足够了。
		顺便提一句,当你写一篇论文时,你可以期望大多数审稿人(和读者)只读一遍。所以需要仔细选择连贯一致的章节和子章节的标题,以及简洁、概括性强的摘要。如果一个评审无法在第一遍阅读理解论文的要点,论文可能会被拒;如果一个读者在超过五分钟的阅读之后无法理解论文的亮点,论文可能再也不会被阅读。因为这些原因,用简单图表概括论文内容的“可视化摘要”是一个很棒的主意,它在科学杂志上越来越多被使用。
		
		\subsection{第二轮}
		在第二遍时,更加仔细地阅读论文,但略过细节(如证明过程)。这有助于阅读论文时,在边栏记下要点或写下注释。Augsburg大学的Dominik Grusemann教授建议:记录下你不理解的内容,或者你想询问作者的问题。如果你是论文评审人,这些注释将在写评审意见时提供帮助,并且在程序委员会期间备份你的评论。
		\begin{enumerate}
			\item 仔细阅读图、表和例证。特别关注图片,坐标轴的标注是否合理?结果是否展示了误差分析,以表明结论在统计上是显著的?诸如此类的常见错误会将匆忙粗制滥造的工作与真正优秀的工作区别开。
			\item 标注出与你研究相关但尚未阅读过的参考文献,留作待读(这是一个了解更多论文背景的好方法)。
		\end{enumerate}
		第二轮阅读对于一个有经验的读者需要1个小时左右。之后,你应该可以掌握论文的内容,并向别人总结出论文的主旨和相应的论据。对于你感兴趣、但不是研究专长的论文,这种程度的细节理解已经足够了。
		不过,有时即使完成了第二轮阅读,你仍然无法理解一篇论文。出现这种情况的原因可能是:你对充满不熟悉的术语和缩写的研究主题不熟悉;亦或是作者使用了你无法理解的证明或实验技术,导致论文大部分内容晦涩难懂;也可能是论文本身写的很烂,包含了未经证实的断言和众多前向的参考引用。当然,还可能是你深夜读文章太累了,精力难以集中理解论文。此时,你可以有这些选择:(a)把论文丢到一边,并寄希望即使看不懂论文也能在工作上获得成功;(b)先把论文暂放,补充相关背景材料知乎再回头阅读;(c)坚持下去,进行第三轮阅读。

		\subsection{第三轮}
		为了完全读懂一篇论文(尤其当你是审稿人时),需要进行第三轮阅读。这一轮的关键点是尝试大体重现论文:即采用与作者相同的假设,重新推演整个工作。通过比较你的推演和作者的论文工作,你不仅可以很容易地发现论文的创新点,也可以发现隐藏的缺陷和未明示的假设。
		这轮阅读更加关注细节。你应当识别出论文的所有假设,并向其发起挑战。更进一步地,你应当站在作者角度设想自己该如何表达一个特定观点。将自己设想和论文的实际论述相比较,能获得关于证明和论述技巧的清晰洞察,并将其纳入自己的工具箱(以提升自己的论证水平)。在此轮阅读中,你也应当记录下对自己未来研究工作有帮助的想法。
		对于新手,这轮阅读需要花费数小时;而即使是有经验的读者,也要花费超过1、2小时。这轮阅读过后,你应当可以仅凭记忆(脱稿)重建出证明论文的结构,同时能清晰地说明论文的优缺点。特别地,你应当能精准地指出隐含假设,缺失的相关工作引用,以及实验和分析方法中存在的潜在问题。	
		
		\section{文献综述}
		文献综述是对研究者论文阅读能力的检验。在这过程中,你需要阅读几十篇、上百篇的论文,而且这些论文可能并不属于你所熟悉的领域。你应该阅读哪些论文?下面介绍如何利用三轮阅读法来回答这个问题。
		第一步,使用Google Scholar或者CiteSeer等学术搜索引擎和一些恰当的关键词,搜到三到五篇近期发表的高引用论文。对这些论文进行第一论阅读,了解大致内容,然后重点阅读论文的“相关工作”(或者文献综述)章节。据此,可以找到近期相关工作的概述,幸运的话,甚至可以发现一篇近期发表的综述文章的索引。如果找到了这样的综述文章,搜索工作就结束了,直接阅读这篇综述就可以了。否则进入第二步。
		第二步,找到上述论文参考文献中反复出现的论文和作者。这些论文及作者 是你所在研究领域的关键文献及学者。先把论文下载好,然后去那些学者的个人网页,查看他们近期将论文发表在哪些期刊、哪些学术会议上。这有助于你找到所在领域的顶级期刊及会议,因为最好的学者通常会将成果发表在最好的期刊或会议上。
		第三步,到上述期刊及会议的网站上查看近期发表的论文。快速浏览论文标题,就能找到近期发表的高质量相关论文。这些论文连同你在第二步中找到的论文就是你进行文献综述时需要阅读的第一批论文。对这些论文进行两轮阅读。如果这些论文都引用了一篇你之前未列入上述名单的论文,那么找到并阅读之。如有必要,这一过程可以反复进行下去。
	
	\section{相关工作}
	如果你是一个论文评审人,你应该再读一读Timothy Roscoe's的论文《writing reviews for systems conferences》\cite{TR}。如果你打算写一篇科技类论文,你应该参考Henning Schulzrinne的的网站\cite{HS}和 Georage Whitesides关于此过程的出色总结\cite{GMW}。最后,Simon Peyton Jones在其站点上分析了整个研究技能图谱\cite{SPJ}。
	来自Psychology Inc.的Iain H. McLean已经收集了可供下载的“评审矩阵”,针对实验心理学领域论文使用三轮阅读方法可以简化论文的评审,并且在最小修改情况下,可以被应用到其他领域的论文阅读和评审上\cite{IHM}。
	
	\section{致谢}
	本文的第一版由我的学生起草,他们是 Hossein Falaki,Earl Oliver 及 Sumair Ur Rahman。感谢他们的工作。我也从 Christophe Doit 敏锐的评论以及 Nicole Keshav 出色的编辑中获益匪浅。
	我将会根据读者的反馈不断更新本文。如果有任何评论或改进建议,请直接给我发电邮。感谢多年来许多读者鼓励和支持的反馈。
%	\section{参考文献}

	\bibliographystyle{plain}
	\bibliography{paper_ref_db}
\end{document}